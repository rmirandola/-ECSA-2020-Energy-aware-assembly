In \cite{networks19} there is a table summarizing the work done in energy saving/efficiency from a software point of view.This can be helpful.


% We should position this work wrt FGCS-2019
In~\cite{Zeng:2018} the energy demand of a server (i.e., fog/edge node) consists of static and dynamic parts. The static part is fixed once the server is activated, regardless of the resource utilization.  While the dynamic part is proportional to the resource utilization (i.e., how many services a physical node hosts)~\cite{Zeng:2018}. In our previous paper~\cite{Dangelo:2020}, we did not aim at measuring the impact of multiple services on single nodes.  Without loss of generality we assumed that each node hosted a single service. Moreover, we did not take into account replication of services. Because of this, if we assume that all the services have the same resource requirement, our computation energy (static  plus dynamic) according to this model is the same for every node of the platform.

\mirko{Differentiate between works on service placement on fog/edge (cite some) and service assembly on fog/edge (should we look more RW respect our journal?). In this work we assume that the placement of services in the infrastructure is done in advance and we aim at assemble them}