
Given the fast growth of smart mobile gadgets, it is expected that a large number of intelligent applications will be deployed at the edge of wireless networks. As such, the 6G wireless network will be designed to leverage advanced wireless communications and mobile computing technologies to support applications at various edge (mobile) nodes~\cite{Letaief:2019}.

\begin{itemize}
    \item to overcome the resource limitation of edge devices, on-device distributed computing provides new opportunities. The capacity and latency of wireless links are the key bottlenecks of mobile applications~\cite{Letaief:2019}
    \item due to the geo-distribution characteristics, different fog/edge nodes shall have different green energy harvesting conditions (e.g., generation rate), different resource capacities and different communication energy costs~\cite{Zeng:2018}
    \item the possibly load-dependent nature of some QoS attributes (e.g., those concerning performance), might rule out simple greedy service selection policies~\cite{Dangelo:2020}
\end{itemize}

\subsection{... commenti ....}
\begin{enumerate}
    \item ogni servizio $S$ pubblicizza la sua QoS $S.Q$; ogni $S$ quindi viene a conoscenza di possibili candidati $S_1, S_2 , \dots $ per ogni sua dipendenza, ognuno con la sua $S_i.Q$ "dichiarata";
    \item $S$ usa il criterio QoS-aware di sez. \ref{sec:service-select} per selezionare il servizio "migliore" $\overline{S}$; 
    \item $S$ usa la sua esperienza diretta della QoS di $\overline{S}$ per aggiornare il suo indice di trust riguardo $\overline{S}$;
    \item $S$ usa espressione \ref{eq:compound_utility2} (p.es. l'istanziazione \ref{eq:reliability-h-function} e \ref{eq:reliability-H-function}) per stimare la sua $S.Q$; DOMANDA: ma il modello in sez \ref{sec:qualitymod} serve veramente? $S$ (meglio: un servizio di monitoraggio locale su $node(S)$) non potrebbe semplicemente stimare $S.Q$ sulla base di osservazioni dirette?
    \item $S$ usa i modelli in sez. \ref{sec:energymod} per stimare i suoi consumi energetici $S.L^{comp}$ e $S.L_n^{comm}$; questi indici non necessitano di informazioni relative a servizi ospitati da altri nodi, quindi infomazione energetica non deve essere fatta circolare, e quindi non serve un modello di trust; DOMANDA: servono questi modelli? $S$ (il monitor di $node(S)$) non potrebbe stimare i consumi energetici direttamente sulla base di osservazioni fatte? RISPOSTA: forse per $S.L^{comp}$ si potrebbero usare solo osservazioni dirette; per $S.L_n^{comm}$ il modello forse serve, per stimare i costi di comunicazione come funzione di banda, latenza e volume di dati da scambiare 
    \item \vincenzo{le considerazioni fatte sopra sono tutte da rivedere ...}
\end{enumerate}
