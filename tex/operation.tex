\section{System Architecture}
\label{sec:architecture}

\vincenzo{da mettere da qualche parte: .... We are assuming that each service $S$ advertises its QoS by publishing the current value  of the QoS index $S.Q$, known through an internal monitoring activity, as described $where???$. Moreover, $S$ could also advertise (see later ....) its system-wide energy consumption by publishing the current values of $S.E^{comp}$ and $S.E_n^{comm}$, $n\in\mathbf{N}$, estimated according to equations (\ref{eq:comp-profile}) and (\ref{eq:comm-profile}), respectively. ... }

The goal is to drive the service-oriented system towards the construction of an energy-efficiency service assembly (see in Section~\ref{sec:welfareIndex}). In this section we present a fully decentralized architecture allowing for achieving this goal. 


\begin{figure}[t]
\centering
\includegraphics[width=0.8\textwidth]{figures/NodeArchitecture.pdf}
\caption{Node Architecture}
\label{fig:nodeArch}
\end{figure}

Figure~\ref{fig:nodeArch} shows the set of components deployed at each $Node_i$: \textit{Service Pool} is the set of services $S \in \mathbf{S}$ running on $Node_i$. \textit{Monitor} is in charge of monitoring the energy consumption (i.e., $S.X^{comp}$ and $S.X^{comm}$ ) for each service $S$ in the Service Pool, and notifying detected changes to the \textit{Assembly Manager}.

%
\textit{Assembly Manager} receives information about the energy consumption of local and remote services from \textit{Monitor} and \textit{Dissemination}, respectively, which are used to build the assembly. In particular, it receives from \textit{Dissemination} the set $S.Prov$ that specifies which services should currently be used to solve the dependencies of a local service $S$, and manages the corresponding bindings. Moreover, it receives notifications of incoming binding requests for each local service $S$, and keeps updated the corresponding set $S.Req$. 

Finally, \textit{Dissemination} implements decentralized information dissemination by exploiting a gossip communication model~\cite{shah2009gossip}, which includes two concurrent threads, namely \textit{ActiveThread} and \textit{PassiveThread} (See Algorithm~\ref{alg:gossip}).

 %an active thread that sends a message to a set of peers, and a passive thread that responds to messages received from other peers.
   
\textit{ActiveThread} sends, every $\Delta t$ time units, a Gossip
message to its {\em peer set}\footnote{The peer set is provided by the underlying gossip communication protocol~\cite{shah2009gossip}.}. The message payload is a set of services, containing the list of currently bound dependencies $S.Prov$ plus $S$ itself.

\textit{PassiveThread} listens for messages coming from other peers. Upon receiving a message containing the set $\mathbf{M}$, it checks all services
$S_k \in \mathbf{M}$ to see whether some of them can be used to fill local services dependencies. If $\sstype{S_k}$ is required as a dependency, then $S_k$ is considered as a candidate to be added to $\mathit{Best}_{S,d}$
(line~\ref{line:is-depBIS}). The decision whether to include $S_k$ in
$\mathit{Best}_{S,d}$ is taken by function \textsc{Update}()
(line~\ref{line:updateBIS}), possibly dropping from $\mathit{Best}_{S,d}$ some
other service whose utility is worse than $S_k$. The update of the sets
$\mathit{Best}_{S,d}$ can lead to a substitution of the service currently used
to solve dependency $d$ (as specified in the set $S.Prov$) with a new
``better'' service taken from $\mathit{Best}_{S,d}$. The decision about this
possible substitution is taken by function $\textsc{Select}()$
(line~\ref{line:selectBIS}), implemented following one of the selection criterion described in Section~\ref{sec:service-select}. 

As is typical with gossip-based protocols, a new instance of
Algorithm~\ref{alg:gossip} is created at each node for each service $S \in \mathbf{S \ B}$ in deployed in the Service Pool.



\begin{algorithm}[t]
\caption{Algorithm executed by the agent responsible for service $S$}\label{alg:gossip}
\begin{small}
\begin{algorithmic}[1]
\Statex \emph{// Input parameters}
\State $S; \textsc{Select}(); \textsc{Update}()$
\Statex \emph{// Local variables}
%\State $i \leftarrow \textsc{GetProcID}()$\label{line:init-startBIS}
\State $S.Prov \leftarrow \emptyset$\label{line:init-startBIS}
\State $\mathit{Best}_{S,d} \leftarrow \emptyset;\ \mbox{for all}\ d \in S.Deps$\label{line:init-endBIS}
\Statex
\Procedure{ActiveThread}{}
\Loop
\State Wait $\Delta t$
\ForAll{$S_j \in \textsc{GetPeers}()$}
\State Send $\langle S.Prov \cup \{S\} \rangle$ to $S_j$\label{line:active-send}
\EndFor
\EndLoop
\EndProcedure
\Statex
\Procedure{PassiveThread}{}
\Loop
\State Wait for message $\langle \mathbf{M} \rangle$ from $S_j$
\ForAll{$S_k \in \mathbf{M}$}

\If{there exists $d\in S.Deps$ s.t. $matches(d,\sstype{S_k})>0$}\label{line:is-depBIS}
\State $\mathit{Best}_{S,d}  \leftarrow \textsc{Update}(\mathit{Best}_{S,d}, S_k)$\label{line:updateBIS}
\State $S.Prov \leftarrow  \textsc{Select}(\mathit{Best}_{S,d}, S.Prov)$\label{line:selectBIS}
\EndIf
\EndFor
\EndLoop
\EndProcedure
\end{algorithmic}
\end{small}
\end{algorithm}
