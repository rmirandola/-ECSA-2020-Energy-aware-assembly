\section{System Architecture}
\label{sec:architecture}

The goal is to drive the service-oriented system we are considering towards the construction of a service assembly that is energy efficient (see in Section~\ref{sec:welfareIndex}). In this section we present a fully decentralized architecture allowing for achieving this goal. 






\begin{figure}[t]
\centering
\includegraphics[width=0.8\textwidth]{figures/NodeArchitecture.pdf}
\caption{Node Architecture}
\label{fig:nodeArch}
\end{figure}

Figure~\ref{fig:nodeArch} shows the set of macrocomponents deployed at each $Node_i$: \textit{Service Pool} is the set of services $S \in \mathbf{S}$ running on $Node_i$; \textit{Monitor} is in charge of monitoring energy consumption (i.e., $S.X^{comp}$ and $S.X^{comm}$ ) for every service $S$ in the Service Pool; \textit{Assembly Manager}   

%, namely
%\textit{Dissemination}, \textit{Assembly Manager}, and 


 hosts as set of offered services (i.e., , plus an instance of these macrocomponents.

%Overall, these pairs, by cooperating among them as outlined below and
%detailed in Section~\ref{subsec:algorithm}, give rise to a fully decentralised
%implementation of the \gmw{} operations. In particular, the {\em Assembly
%Management} macrocomponents cooperate according to a gossip schema that allows
%fully decentralised information dissemination and decision-making about the
%assembly construction and maintenance. This makes the system robust and
%scalable in the presence of events like arrival of new requirements,
%upgrade/downgrade of service utility (including the extreme case of service
%failure), or arrival/departure of new peers (and corresponding hosted services).